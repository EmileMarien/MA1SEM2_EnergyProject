
\section{Energy Bill}
\label{sec:energy-bill}
The total energy bill for a year is made up of the energy cost, network rates and levies. For the energy cost, two actual Belgian methodologies are compared in this report, being the dual day-night tariff and the dynamic tariff. Each of the components is explained in this section. 

\subsection{dual day-night tariff}
\label{sec:dual-day-night}
The dual day-night tariff is a tariff that is based on the time of the day and day of the week and consists of two parts, being the day- and night tariff. Throughout the working week and from 7:00 to 22:00, the day tariff, with a price of 0.171 \euro/kWh in our case is active. Outside those hours and through the weekend, the night tariff, with a price of 0.153 \euro/kWh in our case is valid. This was historically introduced to stimulate the use of electricity during the night and weekends, when the demand is lower. With the introduction of renewable energy sources, this tariff is controversial, as the production of energy became higher during the day and lower during the night, going into the goal of the tariff.
\\ \\
The energy cost is calculated by multiplying the energy consumption from the grid with the corresponding tariff. This consumption value is the net positive power flow from the grid, meaning that the energy produced by the PV panels and coming from the battery is subtracted from the total consumption.
\\ \\
Next to the variable tariff, a tariff undependent of the night or day is added. this is related to the cost of the green energy and CHP used to produce the electricity and has a rate of 0.0156 \euro/kWh.
\\ \\
In case of an additional PV installation, the excess energy produced by the PV panels that is not consumed by the load or battery, is sold back to the grid at a fixed price of 0.03 \euro/kWh. This is subtracted from the total energy bill.

\subsection{Capacity Tariff}
\label{sec:capacity-tariff}
The capacity tariff is part of the network costs and a fixed cost that is calculated based on the maximum power that is drawn from the grid at any given time. This to encourage the consumer to have a more stable power consumption instead of peaks for which a higher system infrastructure capacity is needed.
\\ \\
It is calculated by taking the maximum power drawn from the grid in a 15 minute interval for each month, averaging these and multiplying with the capacity tariff, set at 0.5 \euro/kW. Table \ref{tab:capacity-tariff} shows the peak power drawn from the grid for each month and the corresponding final total capacity tariff cost.

\begin{table}[H]
\centering
\begin{tabular}{|c|c|c|}
\hline
Month & Peak Power [kW] & Capacity Tariff [€] \\ \hline
January & 0.5 & 0.25 \\ \hline
February & 0.5 & 0.25 \\ \hline
March & 0.5 & 0.25 \\ \hline
April & 0.5 & 0.25 \\ \hline
May & 0.5 & 0.25 \\ \hline
June & 0.5 & 0.25 \\ \hline
July & 0.5 & 0.25 \\ \hline
August & 0.5 & 0.25 \\ \hline
September & 0.5 & 0.25 \\ \hline
October & 0.5 & 0.25 \\ \hline
November & 0.5 & 0.25 \\ \hline
December & 0.5 & 0.25 \\ \hline
\end{tabular}
\caption{Capacity Tariff}
\label{tab:capacity-tariff}
\end{table}

\subsection{Consumption costs}
\label{sec:consumption-costs}
Another part of the network costs are the consumption costs. These are attributed to the distribution of the energy and the maintenance of the grid and dependent on the province, Limburg in this case, of the household. These are calculated by taking the total energy consumption from the grid and multiplying it with the consumption tariff, set at 0.02 \euro/kWh. The same happens for injection with a tariff of 0.02 \euro/kWh. 

\subsection{Levies}
\label{sec:levies}
Lastly, the levies are added to the total energy bill. These are costs that are used to finance the renewable energy production and the energy efficiency programs. The levies are calculated by taking the total energy consumption from the grid and multiplying it with the levy tariff, set at 0.05 \euro/kWh. 

\subsection{Total Energy Bill}
\label{sec:total-energy-bill}
The total energy bill is the sum of the energy cost, capacity tariff, consumption costs and levies. The chosen rates and an example for our house with 14 PV panels, of which the power flow to the grid is given in table \ref{tab:power-flow-to-grid}, are shown in table \ref{tab:total-energy-bill_dual}. With everything except the energy cost being equal for the dynamic tariff, the total energy bill for the dynamic tariff is ... euro.



\begin{table}[H]
    \centering
    \begin{tabular}{|c|c|c|}
    \hline
     & \textbf{Peak} & \textbf{Off-peak} \\ \hline
    \textbf{Injection} & 1000 & 0 \\ \hline
    \textbf{Consumption} & 1000 & 0 \\ \hline
    \end{tabular}
    \caption{Power flow to the grid}
    \label{tab:power-flow-to-grid}
\end{table}

\begin{table}[H]
\centering
\begin{tabular}{|c|c|c|}
\hline
\textbf{Component} & \textbf{Rate} & \textbf{Cost} \ \hline
\textbf{Energy cost} & & \ \hline
Energy Cost Day & 0.171 \euro/kWh & 0.171 \euro/kWh \\ \hline
Energy Cost Night & 0.153 \euro/kWh & 0.153 \euro/kWh \\ \hline
Energy Injection & 0.03 \euro/kWh & 0.03 \euro/kWh \\ \hline
Green Energy Cost & 0.0156 \euro/kWh & 0.0156 \euro/kWh \\ \hline
\textbf{Network rates} & & \ \hline
Injection Tariff & 0.03 \euro/kWh & 0.03 \euro/kWh \\ \hline
Capacity Tariff & 0.5 \euro/kW & 3 \euro \\ \hline
Consumption Tariff & 0.02 \euro/kWh & 0.02 \euro/kWh \\ \hline
\textbf{Levies} & & \ \hline
Levy Tariff & 0.05 \euro/kWh & 0.05 \euro/kWh \\ \hline
\end{tabular}
\end{table}


\section{Impact Assessment of EV}
\label{Impact Assessment of EV}
The impact of the addition of an electric vehicle (EV) to the system is assessed 5 years after the installation of the PV system. They are three main aspects which will have an impact on the system. First, the overall energy usage and the capacity of the EV charger are discussed. Then, the charging behavior is discussed and a scenario with and without smart charging is analyzed. 
\subsection{Energy usage and charger size}
First, for the overall energy usage of the EV vehicle, certain assumptions need to be made. First, the average Belgian drives $35km$ daily. To take the worst case scenario, assume that the average daily distance driven is $70km$. Further, the chosen car is a Porsche Taycan. This car has a usable battery capacity of $82.3kWh$ and an efficiency of $0.17kWh/km$. With this data an additional yearly energy usage can be calculated which amounts to $4343.5kWh$. Further, the EV has a range of $485km$. Assume the worst case scenario cold temperature and use of air conditioning, This causes the range to drop to $405km$. \\ \\
Next, there are multiple options considering the type of charger that can be installed. The possibilities are a single-phase $3.7kW$ charger, a single-phase $7.4kW$ charger and a three-phase $11kW$ charger. For this analysis, it is assumed that the charger used to charge the EV is a type 2 single-phase with a charging power of $3.7kW$. 
\subsection{Charging behavior}
The last aspect is the charging behavior. For this analysis, it is assumed that the EV can only be charged at home. Further, the EV is never charged past 80\%. This is done because charging efficiency decreases substantially when above this marker, and the long-term health of the EV battery is maintained. The two scenarios that are analyzed are one without smart charging and one with smart charging. 
\subsubsection{Without smart charging}
In the scenario without smart charging, the EV is charged at a fixed time every day, independent of what the existing energy production, load, or battery charge is. Further, the EV charger is then used at full capacity. 
\subsubsection{With smart charging}
In the next scenario, smart charging is applied. This means that the EV will be charged optimally on moments when there is a combination of a low load with a high PV production. In this way, the EV is not charged from the grid when prices are high. Further, moments when the battery charge is high are also optimal to charge. In the graph below, the yearly average net production, and battery charge for a day are given. 
\subsubsection{Back to grid charging}
 In the last scenario, the EV can be used as a battery and thus deliver energy back to the grid. For this scenario, some assumptions for the availability of the EV are made, which are listed in Table \ref{tab:EV-availability}. When the houseowner is out for work or dinner, the EV is not connected to the house and thus cannot function as a battery. Instead its battery is used for driving and reduces with an average of 1.5 kWh each hour. Next, the car only charges when prices are low, being overnight during the workweek and during the day in the weekend. Outside those periods, the car can deliver electricity to the house while it is guaranteed that the car has enough energy to drive during the next period. A typical flow of energy to the EV and variation of the battery charge is given in Figure \ref{fig:EV-availability}.

\begin{table}[H]
\centering
\begin{tabular}{|c|c|}
\hline
\textbf{Availability} & \textbf{Value} \\ \hline
Max charging power & 3.7 kW \\ \hline
Max discharging power & 3.7 kW \\ \hline
Battery capacity & 82.3 kWh \\ \hline
Battery usage when not connected & 1.5 kWh/h \\ \hline
\end{tabular}
\caption{EV availability}
\label{tab:EV-availability}
\end{table}

\begin{figure}[H]
    \centering
    \includegraphics[width=0.8\textwidth]{images/EV-availability.png}
    \caption{Energy flow to the EV and battery charge}
    \label{fig:EV-availability}
\end{figure}




